\documentclass[a4paper,12pt]{article}
\usepackage[brazil]{babel}
\usepackage[utf8]{inputenc}
\usepackage{amsthm}
\usepackage{amsfonts}
\begin{document}

\title{Condição de Imagem com Hadoop}
\author{Ian Liu Rodrigues -- 061485}
\maketitle

\begin{abstract}
  A \emph{condição de imagem} é uma etapa do algoritmo \emph{Migração em
  Tempo Reverso}, ou \emph{MTR}, cujo objetivo é criar um mapa de
  coerência de sinais coletados em campo.
\end{abstract}

\section{Introdução}

\subsection{Cubo de dados}

Um cubo de dados de tamanho $n_z \times n_x \times n_t$ representa um
filme da evolução de ondas acústicas se propagando em uma malha de
tamanho $n_z \times n_x$. Cada fatia do dado no plano $XZ$ representa
amplitudes de pressão num dado tempo $t$. Em disco, um cubo de dados
pode chegar a ocupar alguns gigabytes.

Os cubos de dados são gerados por meios de simulações acústicas. O
método das diferenças finitas pode ser usado para realizar tais
simulações.

\subsection{Fonte}

Uma fonte sísmica pode ser um explosivo que gera grandes pressões no
solo. Essas ondas de pressão se propagam na estrutura rochosa e são
refletidas quando encontram mudanças de solo, rupturas, etc.

Uma das entradas para o problema da \emph{condição de imagem} é o cubo
de dados gerado por uma fonte.

\subsection{Receptor}

Um receptor é um sismógrafo que coleta vibrações na superfície do solo.
No algoritmo \emph{MTR} essas vibrações são usadas para gerar vários
cubos de dados, um para cada receptor. Estes cubos também são dados de
entrada para calcular a condição de imagem.

\section{Condição de Imagem}

A condição de imagem consiste em realizar uma multiplicação
ponto-a-ponto de cada cubo de dados gerados pelos receptores com o cubo
de dados gerado pela fonte. Após este procedimento, somamos todos esses
resultados em um só

\section{Exemplo}

\end{document}
